\section{Auswertung}
\label{sec:Auswertung}
\subsection{Bestimmung der Charakteristik des Geiger-Müller-Zählrohrs}
Die durch die Messung erhaltenen Messwerte sind in der Tabelle \ref{tab:mess} aufgetragen.
\begin{table}
    \centering
    \caption{Spannung, Aktivität des $\beta$-Strahlers und Zählerstrom.}
    \label{tab:mess}
    \begin{tabular}{S[table-format=3.0] S[table-format=5.0] S[table-format=1.2(0)e0]}
        \toprule
        {$U/\si{\volt}$} & {Counts} & {$I/\si{\micro\ampere}$}  \\
        \midrule
        320 & 13914 & 0.2 \\
        330 & 14186 & 0.3 \\
        340 & 14550 & 0.3 \\
        350 & 14414 & 0.4 \\
        360 & 14448 & 0.4 \\
        370 & 14703 & 0.5 \\
        380 & 14753 & 0.5 \\
        390 & 14587 & 0.6 \\
        400 & 14662 & 0.6 \\
        410 & 14819 & 0.7 \\
        420 & 14822 & 0.7 \\
        430 & 14820 & 0.8 \\
        440 & 15033 & 0.9 \\
        450 & 14903 & 0.95 \\
        460 & 15125 & 1.0 \\
        470 & 15013 & 1.0 \\
        480 & 15076 & 1.1 \\
        490 & 14963 & 1.1 \\
        500 & 14790 & 1.2 \\
        510 & 15127 & 1.2 \\
        520 & 15237 & 1.3 \\
        530 & 15098 & 1.4 \\
        540 & 15076 & 1.4 \\
        550 & 15163 & 1.45 \\
        560 & 14876 & 1.5 \\
        570 & 15207 & 1.6 \\
        580 & 15064 & 1.6 \\
        590 & 15124 & 1.7 \\
        600 & 15097 & 1.7 \\
        610 & 15080 & 1.8 \\
        620 & 15113 & 1.8 \\
        630 & 15356 & 1.9 \\
        640 & 15473 & 2.0 \\
        650 & 15406 & 2.0 \\
        660 & 15472 & 2.1 \\
        670 & 15856 & 2.1 \\
        680 & 16136 & 2.2 \\
        690 & 16534 & 2.3 \\
        700 & 17078 & 2.4 \\
        \bottomrule
    \end{tabular}
\end{table}
\noindent Die aufgenommenen Zählraten sind Poissonverteilt, daher ergibt sich ihre Unsicherheit wie folgt:
\begin{equation}
  \sigma = \sqrt{N}
\end{equation}
Die Messwerte sind in Abbildung \ref{fig:plot} aufgetragen und es wird eine lineare Ausgleichsgerade durch diese gezogen.
\begin{figure}[H]
  \centering
  \includegraphics{build/messung1.pdf}
  \caption{Charakeristik des Zählrohrs.}
  \label{fig:plot}
\end{figure}
\noindent  Die Ausgleichsgerade wird mit Python/SciPy mit der Funktion $f(x)= Ax + B$ erstellt.
Damit erfolgt eine Steigung von
\begin{equation}
  A = \SI{4.9 \pm 0.5}{\percent\per\num{100}\volt}
\end{equation}
und ein $y$-Achsenabchnitt von
\begin{equation}
  B =   \SI[per-mode=reciprocal]{224.6\pm 2.4}{\volt}        .
\end{equation}
Die Steigung des Plateaus kann auch in $\si{\percent\per{100\volt}} angegeben werden:
\begin{equation}
	A = \SI{2.03\pm0.19}{\precent\per{100\volt}}
\end{equation}
\subsection{Oszillographische Messung der Totzeit}
Aus dem Bild des Oszilloskops wird eine Totzeit von
\begin{equation*}
  t_\text{tot} = \SI{70}{\micro \second}
\end{equation*}
und eine Erholungszeit von
\begin{equation*}
    t_\text{erhol} = \SI{225}{\micro\second}
\end{equation*}
bestimmt.
\subsection{Bestimmung der Totzeit mit der Zwei-Quellen-Methode}
Die erhaltenen Messwerte sind in Tabelle \ref{tab:2strahl} aufgetragen.
\begin{table}[H]
  \caption{Gemessene Werte für die Zwei-Quellen-Methode.}
  \label{tab:2strahl}
  \centering
  \sisetup{table-format=5.3(4)}
  \begin{tabular}{c S[table-format=5.0(0)] S}
    \toprule
    {Quelle} & {Counts} & {Aktivität$/\si{\becquerel}$}\\
    \midrule
    {$N_1$}     & 14440 & 240.667\pm2.003  \\
    {$N_{1+2}$} & 14987 & 249.783\pm2.040  \\
    {$N_2$}     & 614   & 10.233 \pm0.413  \\
    \bottomrule
  \end{tabular}
\end{table}
Damit lässt sich die Totzeit mit Gleichung\eqref{eqn:totzeit} auf
\begin{equation*}
  T= \SI{226.77\pm583.92}{\micro\second}
\end{equation*}
abschätzen.
Der Fehler berechnet sich hier nach Gauß mit:
\begin{align*}
  \symup{\Delta} T ^2 = & \left(\frac{2N_1 N_2 - 2N_2(N_1 + N_2 - N_{1+2})}{2 N_1 N_2}^2\right)^2 \symup{\Delta} N_{1}^2 \\
  & + \left(\frac{2N_1 N_2 - 2N_1(N_1 + N_2 - N_{1+2})}{2 N_1 N_2}^2\right)^2\symup{\Delta} N_{2}^2 \\
  & +\left(\frac{1}{2 N_1 N_2}\right)^2 \symup{\Delta} N_{1+2}^2\\
\end{align*}
%
\subsection{Bestimmung der mittleren Ladung pro Event}
Die Ladung wird gemäß Formel \eqref{eqn:ladung} berechnet und ist gegen die entsprechende Spannung in
Tabelle \ref{tab:ladung} aufgetragen.
\begin{table}
    \centering
    \caption{Spannung und Landung pro Event.}
    \label{tab:ladung}
    \begin{tabular}{S[table-format=3.0] S[table-format=2.2e2]}
        \toprule
        {$U/\si{\volt}$} & {$\symup{\Delta}q/e$}  \\
        \midrule
320 &    5.38e12\\
330 &    7.92e12\\
340 &    7.72e12\\
350 &    10.39e12\\
360 &    10.37e12\\
370 &    12.74e12\\
380 &    12.69e12\\
390 &    15.41e12\\
400 &    15.33e12\\
410 &    17.69e12\\
420 &    17.69e12\\
430 &    20.22e12\\
440 &    22.42e12\\
450 &    23.87e12\\
460 &    24.76e12\\
470 &    24.95e12\\
480 &    27.33e12\\
490 &    27.53e12\\
500 &    30.39e12\\
510 &    29.71e12\\
520 &    31.95e12\\
530 &    34.73e12\\
540 &    34.78e12\\
550 &    35.82e12\\
560 &    37.77e12\\
570 &    39.41e12\\
580 &    39.78e12\\
590 &    42.10e12\\
600 &    42.17e12\\
610 &    44.71e12\\
620 &    44.61e12\\
630 &    46.34e12\\
640 &    48.41e12\\
650 &    48.62e12\\
660 &    50.83e12\\
670 &    49.60e12\\
680 &    51.06e12\\
690 &    52.10e12\\
700 &    52.63e12\\
        \bottomrule
    \end{tabular}
\end{table}
