\section{Diskussion}
\label{sec:Diskussion}
In guter näherung lässt sich ein Plateau für die Charakterristik des Geiger-Müller-Zählrohrs erkennen.
Die große Differenz der bestimmten Totzeiten lässt sich zum einen damit erklären, dass beide Methoden lediglich eine Abschätzung sind und keine genauen Werte versprechen.
Andereseits konnte, durch die Eichung des Oszilloskops der Wert stark verändert werden, somit ist die Aussagekraft dieses Wertes stark anzuzweifeln.
Anhand von Tabelle \ref{tab:ladung} lässt sich erkennen, dass die Ladungsmenge mit der anliegenden Spannung ansteigt.
Es kommen Ladungen mit dem Faktor $\num{10e12}$ im Zählrohr an.
Dies entspricht nicht mehr dem Geiger-Müller-Bereich.
Das Zählrohr müsste sich eigentlich entladen.
Anhand der großen Anzahl lässt sich folgern, dass es sich nicht um lokale Entladungen handelt sondern diese im ganzen Zählrohr stattfinden.
