\section{Durchführung}
\label{sec:Durchführung}
Der Veruch wird gemäß Abbildung \ref{fig:aufb} aufgebaut.
\begin{figure}[H]
  \centering
  \includegraphics[scale=0.4]{content/Aufbau.png}
  \caption{Versuchsaufbau.}
  \label{fig:aufb}
\end{figure}
\noindent Das Zählrohr wird an einen Widerstand angeschlossen.
An dem Widerstand entsteht ein Spannungsimpuls, welcher durch einen Kondensator ausgekoppelt wird.
Dieser wird am Verstärker vergrößert und vom Zähler registriert.

\subsection{Aufnahme der Charakteristik des Zählrohrs}
Der Strahlengang eines $\beta$-Strahlers wird auf das Zählrohr ausgerichtet.
Die Messung wird mit einer Anliegenden Spannung $U= \SI{300}{\volt}$ begonnen.
Die Aktivität wird für $\SI{60}{\second}$ aufgenommen.
Zusätzlich wird mit einem empfindlichen Strommessgerät der Zählerstrom gemessen und über die Messzeit gemittelt.
Die Anliegende Spannung wird nach jeder so erfolgten Messung um $\SI{10}{\volt}$ erhöht bis eine Spannung von $\SI{700}{\volt}$ erreicht wird.
\subsection{Oszillographische Messung der Totzeit und der Erholungszeit}
Das Zählrohr wird an ein Oszilloskop angeschlossen.
Die Zählrohrspannung wird auf $\SI{700}{\volt}$ hochgestellt.
Aus dem so enstantendem Bild wird die Totzeit und die Erholungszeit abgelesen.
\subsection{Bestimmung der Totzeit mit der Zwei-Quellen-Methode}
Die Zählrohrspannung wird auf $500$ Volt geschaltet.
Die Aktivität des ausgerichteten Strahlers wird für $\SI{60}{\second}$ gemessen.
Ein zweiter, schwächerer, $\beta$-Strahler wird auf  das Zählrohr ausgerichtet.
Erneut wird die Aktivität für $\SI{60}{\second}$ gemessen.
Der erste $\beta$-Strahler wird entfernt und die Aktivität des zweiten Strahlers wird nach gleichem Verfahren gemessen.
